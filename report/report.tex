\documentclass[a4paper,12pt]{article}
\usepackage[hidelinks]{hyperref}
\usepackage{float}
\usepackage{graphicx}
\usepackage{listings}
\usepackage[utf8]{inputenc}
\usepackage{etoolbox}
\usepackage{fullpage}
\renewcommand*\contentsname{Indice}
\renewcommand*\figurename{Fig.}
\usepackage{setspace}
\usepackage{parskip}
\usepackage{subfigure}
\usepackage{amsmath}


\makeatletter
\patchcmd\l@section{%
  \nobreak\hfil\nobreak
}{%
  \nobreak
  \leaders\hbox{%
    $\m@th \mkern \@dotsep mu\hbox{.}\mkern \@dotsep mu$%
  }%
  \hfill
  \nobreak
}{}{\errmessage{\noexpand\l@section could not be patched}}
\makeatother

\setcounter{secnumdepth}{0}

% un po' di estetica...
\usepackage{fancyhdr}
\pagestyle{fancy}
\setlength{\headsep}{0.35in}
\let\MakeUppercase\relax

% blocchi di codice
\usepackage{listings}
\lstset{
	breaklines=true, 
	frame=single, 
	numbers=left,
	tabsize=2,
	basicstyle=\scriptsize,
	showstringspaces=false
}

\setlength{\parindent}{2em}
\setlength{\parskip}{0.5em}
\renewcommand{\baselinestretch}{1.5}

\fancyhf{} % clear all fields
\fancyfoot[C]{\thepage}

\frenchspacing

\begin{document}

\begin{titlepage}
\noindent
    \vspace*{5mm}
	\begin{minipage}[t]{0.15\textwidth}
	    \vspace*{5mm}
		\vspace{-3.5mm}{\includegraphics[scale=1.8]{img/logo_bicocca.png}}
	\end{minipage}
	\hspace{1cm}
	\begin{minipage}[t]{0.9\textwidth}
	      \vspace*{5mm}
		{
			\setstretch{1.42}
			{\textsc{Università degli Studi di Milano - Bicocca} } \\
			\textbf{Scuola di Scienze} \\
			\textbf{Dipartimento di Informatica, Sistemistica e Comunicazione} \\
			\textbf{Corso di Laurea Magistrale in Informatica} \\
			\par
		}
	\end{minipage}
	
	\vspace{42mm}

\begin{center}
    {\LARGE{
	    	\setstretch{2}
            \textbf{
            	Tecniche di Record Linkage \\ }
    }}        
\end{center}

\vspace{40mm}
	
	
	\begin{flushright}
		\setstretch{1.3}
		\large{Alberici Federico - 808058\\} 
		\large{Bettini Ivo Junior - 806878\\} 
		\large{Cocca Umberto - 807191\\} 
		\large{Traversa Silvia - 816435} 
	\end{flushright}
	
	\vspace{15mm}
	\begin{center}
		{\large{\bf Anno Accademico 2019 - 2020}}
	\end{center}


\renewcommand{\baselinestretch}{1.5}

\end{titlepage}

\tableofcontents

\newpage

\section{Ricerca} 

\subsection{Record Linkage - introduzione}

\subsection{Metodologie (?) di Record Linkage}
% secondo me ci sta vedere anche le slide dei prof ma meglio citare articoli

\subsection{Tools usati}

\subsubsection{Python Record Linkage Toolkit }
Python Record Linkage Toolkit è una libreria che permette di effettuare record linkage sia in una sola fonte di dati che in multiple. Il package contiene metodi di indexing, come blocking e sorted neighbourhood indexing, funzioni per il confronto con diverse misure di similarità possibili e diversi algoritmi di classificazione, sia supervisionati che non. 


\section{Esperimenti}
\subsection{Dataset}
Il dataset scelto per effettuare i nostri esperimenti contiene elenchi di ristoranti di Manhattan provenienti da 12 siti web, presi settimanalmente da Gennaio a Marzo 2009. Sono riportati in tutti il nome del ristorante, l'indirizzo e la città.

\subsection{Pulizia dei dati}
Prima di applicare le tecniche di record linkage è stata effettuata una pulizia dei dati. Avevamo a disposizione 7 file .txt, uno per ogni settimana considerata, contenente dati appartenenti a tutti i siti web. In totale erano presenti 215555 record, suddivisi come illustrato nella tabella:
\begin{table}[H] \centering
\begin{tabular}{|l|l|}
\hline
\multicolumn{1}{|c|}{\textbf{file}} & \multicolumn{1}{c|}{\textbf{records}} \\ \hline
restaurants\_2009\_1\_22.txt & 30401 \\ \hline
restaurants\_2009\_1\_29.txt & 30775 \\ \hline
restaurants\_2009\_2\_05.txt & 30805 \\ \hline
restaurants\_2009\_2\_12.txt & 30863 \\ \hline
restaurants\_2009\_2\_19.txt & 30876 \\ \hline
restaurants\_2009\_2\_26.txt & 30898 \\ \hline
restaurants\_2009\_3\_12.txt & 30937 \\ \hline
\end{tabular}
\caption{Record presenti nei file txt forniti}
\label{tab:Tab}
\end{table}

In particolare, per ogni ristorante è presente il seguente numero di record:

\begin{table}[H]\centering
\begin{tabular}{|c|c|}
\hline
\textbf{restaurant} & \textbf{records} \\ \hline
ActiveDiner & 6184 \\ \hline
DiningGuide & 814 \\ \hline
FoodBuzz & 2079 \\ \hline
MenuPages & 13143 \\ \hline
NewYork & 1774 \\ \hline
NYMag & 5124 \\ \hline
NYTimes & 3095 \\ \hline
OpenTable & 1539 \\ \hline
SavoryCities & 4536 \\ \hline
TasteSpace & 3635 \\ \hline
TimeOut & 14007 \\ \hline
VillageVoice & 2684 \\ \hline
\end{tabular}
\caption{Record per ristorante}
\label{tab:my-table}
\end{table}

\subsubsection{Risultati}
%info sui dati puliti

\subsection{Record Linkage}

\subsection{Gold Standard}
%possiamo vedere il gold standard e confrontarlo con i nostri risultati
\end{document}